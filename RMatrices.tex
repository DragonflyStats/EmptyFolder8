

\section{R Workshop : Matrices}

Creating a matrix

Accessing Rows and Columns

Addition and subtractions

Diagonals and the Identity Matrix

Linear Algebra Functions

Eigenvalues and Eigenvectors

More on Matrices

Using rbind() and cbind()

Solving a System of Linear Equations

 

\subsection{Creating a matrix}


Matrices can be created using the matrix() command. 


The arguments to be supplied are 


    1) vector of values to be entered

    2) dimensions of the matrix, specifying either the numbers of rows or columns.


Additionally you can specify if the values are to be allocated by row or column. 

By default they are allocated by column.



\begin{framed}
\begin{verbatim}
Vec1 = c(1,4,5,6,4,5,5,7,9)             # 9 elements

A = matrix(Vec1,nrow=3)                 #3 by 3 matrix. Values assigned by column.

A

C= matrix(  c(1,6,7,0.6,0.5,0.3,1,2,1), ncol=3 , byrow =TRUE)

C                                               #3 by 3 matrix. Values assigned by row.

 \end{verbatim}
\end{framed}
%===========================================================================%



\section{Accessing Rows and Columns}

Particular rows and columns can be accessed by specifying the row number or column number, leaving the other value blank.

\begin{framed}
\begin{verbatim}
A[1,]   # access first row of A
C[,2]   # access second column of C
\end{verbatim}
\end{framed}
%===========================================================================%
\subsection{Addition and subtractions}

\begin{itemize}
\item For matrices, addition and subtraction works on an element- wise basis. 
\item The first elements of the respective matrices are added, and so on.
\end{itemize}

%----------------------%

\begin{framed}
\begin{verbatim}
A+C                                             
A–C
\end{verbatim}
\end{framed}
%========================================================================%
\subsection{Matrix Multiplication}



To multiply matrices, we require a special operator for matrices; “%*%”. 


If we just used the normal multiplication, we would get an element-wise multiplication.
%========================================================================%
\subsection{Basic Matrix Calculations}


1) Inverting a matrix 


To invert a matrix we use the command solve() with no additional argument.


We can use this same command to solve a system of linear equations Ax=b

We would specify the vector b as the additional argument.



2) Computing the determinant


To compute the determinant, the command is simply \texttt{det()}



3) Compute the transpose 


To compute the transpose of matrix A, we use the command \texttt{t()}.

%----------%

4) Determining the dimensions 


To find the dimensions of matrix A, we use the \texttt{dim()} command


5) Cross Products


We can compute cross products using the \texttt{crossprod()} command. 

\begin{framed}
\begin{verbatim}
solve(A)                            # Inverse of Matrix A    
t(A)                                  # tranpose of A
det(A)                               # determinant of A  
det(t(A))                           # determinant of the transpose of A
A %*% C                           # Matrix Multiplication
C * A                                #Casewise Multiplication
\end{verbatim}
\end{framed}
%------------------------------------------------------%
\subsection{Diagonals and the Identity Matrix}

The \texttt{diag()} command is a very versatile function for using matrices.

It can be used to create a diagonal matrix with elements of a vector in the principal diagonal. 

For an existing matrix, it can be used to return a vector containing the elements of the principal diagonal. 

Most importantly, if k is a scalar, diag() will create a k x k identity matrix.

\begin{framed}
\begin{verbatim}

Vec2=c(1,2,3)

diag(Vec2)     #       Constructs a diagonal matrix based on values of Vec2

diag(A)          #        Returns diagonal elements of A as a vector


diag(3)          #       creates a 3 x 3 identity matrix


diag(diag(A)) #        Creates the diagonal matrix D of matrix A ( Jacobi Method)

\end{verbatim}
\end{framed}
%=========================================%
\subsection{Linear Algebra Functions}


R supports many import linear algebra functions such as cholesky decomposition, trace, rank, eigenvalues etc.


The required results may be determinable from the output of a command that pertains to an overall approach.

\begin{framed}
\begin{verbatim}
eigen(A)       #eigenvalues and eigenvectors       
qr(A)            #returns Rank of a matrix
svd(A)
\end{verbatim}
\end{framed}
%=========================================%
\subsection{Eigenvalues and Eigenvectors}
The eigenvalues and eigenvectors can be computed using the \texttt{eigen()} function.  A data object is created.
This is a very important type of matrix analysis, and many will encounter it again in future modules.

\begin{framed}
\begin{verbatim}

Y = eigen(A)
names(Y)

#   y$val are the eigenvalues of A

#   y$vec are the eigenvectors of A 

\end{verbatim}
\end{framed}
%=========================================%
 




More on Matrices

 

Note that the following commands are useful for Experimental Design.

 
\begin{framed}
\begin{verbatim}

rowMeans(A)                                         Returns vector of row means.

rowSums(A)                                             Returns vector of row sums. 

colMeans(A)                                             Returns vector of column means. 

colSums(A)                                               Returns vector of coumn means. 
\end{verbatim}
\end{framed}
 
%------------------------------------------------------------------------------%



Using rbind() and cbind()

Another methods of creating a matrix is to “bind” a number of vectors together, either by row or by column. 

The commands are rbind() and cbind() respectively.

\begin{framed}
\begin{verbatim}
> x1 =c(1,2) ; x2 = c(3,8)                                                


> D= rbind(x1,x2)


> E = cbind(x1,x2)


> det(D)


[1] 2


> det(E)


[1] 2

 \end{verbatim}
 \end{framed}
 %==========================================%
\subsection{Solving a System of Linear Equations}


To solve a system of linear equations in the form Ax=b , where A is a square matrix, and b is a column vector of known values, we use the \texttt{solve()} command to determine the values of the unknown vector x.

\begin{framed}
\begin{verbatim}
b=vec2  # from before
solve(A, b) 
\end{verbatim}
\end{framed}
%======================================%
\end{document}



