\section{Central Limit Theorem}

Hypothesis testing and confidence interval construction are based on the \textbf{\textit{Central Limit Theorem}}.

Recall the Dice experiment in previous classes.

Dice1=floor(runif(50,min=1,max=7))  

A simple demonstration of the central limit theorem is given by the problem of rolling a large number of dice repeatedly. 
The distribution of the sum (or average) of the rolled numbers will be well approximated by a normal distribution, the parameters of which can be determined empirically.
%---------------------------------------------------% 
\begin{verbatim} 
N=100            #number of loops
Avgs=numeric(N)  #array “Avgs” store the sample means
for( i in 1:N)
     { 
     Dice=floor(runif(50,min=1,max=7));
     Avgs[i]=mean(Dice)  
}                                 
Avgs            #print Avgs dataset to screen
\end{verbatim}
%---------------------------------------------------%
Lets look at the distribution of the means. Are they normally distributed?

\begin{verbatim}
mean(Avgs)          #compute the mean. 
qqnorm(Avgs)        #draws a QQ plot
qqline(Avgs)        #adds trend line to QQplot.
shapiro.test(Avgs)  #Shapiro Wilk test. 
\end{verbatim}

%---------------------------------------------------%
\end{document}
