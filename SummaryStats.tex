%------------------------------------------------------------------------------------------------%

\chapter{Descriptive Statistics}

\section{Basic Statistics}

\footnotesize
\begin{myindentpar}{1cm}
\begin{verbatim}
> X=c(1,4,5,7,8,9,5,8,9)
> mean(X);median(X)       #mean and median of vector
[1] 6.222222
[2] 7
> sd(X)                   #standard deviation of Vector
[1] 2.682246
> length(X)               #sample size of vector
[1] 9
> sum(X)
[1] 56
> X^2
[1]  1 16 25 49 64 81 25 64 81
> rev(X)
[1] 9 8 5 9 8 7 5 4 1
> sort(X)                 #items in ascending order
[1] 1 4 5 5 7 8 8 9 9
> X[1:5]
[1] 1 4 5 7 8
\end{verbatim}
\end{myindentpar}
\normalsize


\section{Summary Statistics}
The \texttt{R} command \texttt{summary()} returns a summary statistics for a simple dataset.
The \texttt{R} command \texttt{fivenum()} returns a summary statistics for a simple dataset, but without the mean.
Also, the quartiles are computed a different way.

\footnotesize
\begin{myindentpar}{1cm}
\begin{verbatim}
> summary(mtcars$mpg)
   Min. 1st Qu.  Median    Mean 3rd Qu.    Max.
  10.40   15.43   19.20   20.09   22.80   33.90 
>
> fivenum(mtcars$mpg)
[1] 10.40 15.35 19.20 22.80 33.90
\end{verbatim}
\end{myindentpar}
\normalsize




\section{Bivariate Data}
\footnotesize \begin{myindentpar}{1cm}
\begin{verbatim}
> Y=mtcars$mpg
> X=mtcars$wt
>
> cor(X,Y)          #Correlation
[1] -0.8676594
>
> cov(X,Y)          #Covariance
[1] -5.116685
\end{verbatim}
\end{myindentpar}\normalsize
%======================================================================%

\section{Histograms}
Histograms can be created using the \texttt{hist()} command.
To create a histogram of the car weights from the Cars93 data set
\footnotesize
\begin{myindentpar}{1cm}
\begin{verbatim}
hist(mtcars$mpg, main="Histogram of MPG (Data: MTCARS) ")
\end{verbatim}
\end{myindentpar}\normalsize
\texttt{R} automatically chooses the number and width of the bars. We can
change this by specifying the location of the break points.
\footnotesize
\begin{myindentpar}{1cm}
\begin{verbatim}hist(Cars93$Weight, breaks=c(1500, 2050, 2300, 2350, 2400,
2500, 3000, 3500, 3570, 4000, 4500), xlab="Weight",
main="Histogram of Weight")
\end{verbatim}
\end{myindentpar}\normalsize



\section{Boxplot}
Boxplots can be used to identify outliers.

By default, the \texttt{boxplot()} command sets the orientation as vertical. By adding the argument \texttt{horizontal=TRUE}, the orientation can be changed to horizontal.
\footnotesize
\begin{myindentpar}{1cm}
\begin{verbatim}
boxplot(mtcars$mpg, horizontal=TRUE, xlab="Miles Per Gallon",
main="Boxplot of MPG")
\end{verbatim}
\end{myindentpar}\normalsize

\begin{figure}
  % Requires \usepackage{graphicx}
  \includegraphics[scale=0.4]{MTCARSboxplot.png}\\
  \caption{Boxplot}\label{boxplot}
\end{figure}


